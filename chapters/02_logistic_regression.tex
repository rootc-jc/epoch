%!TEX root = ../main.tex

\chapter{Logistic Regression}
\label{chp:logistic}

\section{Introduction}
As discussed in the previous chapter, we will build a model which will predict
the outcome based on the model's learning from training dataset. Logistic Regression
is used for predicting discrete values (only 0 an 1 here) unlike continuos values in Linear Regression. It simply
calculates the probabilty of the value, calculated from Linear Regression, lying on the extremes and
output it as 0 and 1.

\section{Model}
The model is simply a little additional calculation on Linear Regression. We use sigmoid function
to restrict the outcome between 0 and 1. If the outcome is close to 1 then we treat it as 1 and 0 otherwise.

\subsection{Sigmoid Function}

For restricting the real number value from the outcome of Linear Regression
between 0 and 1, we use Sigmoid function. It takes real numbers as input and 
outputs continuos values between 0 and 1.

\begin{equation}
    Sigmoid(x) = \frac{1}{1+e^{-x}}
\end{equation}

The graph of Sigmoid function is shown in figure \ref{fig:sigmoid}

\begin{figure}
    \includegraphics[]{figs/sigmoid_plot.png}
    \caption{Sigmoid function graph}
    \label{fig:sigmoid}
\end{figure}

\subsection{Mathematical Understanding}